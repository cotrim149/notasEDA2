\documentclass{article}

\usepackage[utf8]{inputenc}

\title{Aulas de Estruturas de Dados e algoritmos 2}
\author{cotrim149}

\begin{document}
\maketitle

\section{Busca Sequencial}
	\begin{enumerate}
	\item Complexidade média(Tempo de demora para resposta): n/2
	\item O(n)
	\item Métodos para otimização
		\begin{itemize}
		\item Sentinela: Consiste em adicionar um elemento de valor x no final do vetor.
		\end{itemize}		 
	\item Alternativa: Lista encadeada
	\item Aumento de eficiência
		\begin{itemize}
		\item Método mover para frente: Sempre que uma pesquisa obter êxito, o registro recuperado é colocado no ínicio da lista. \textbf{Desvantagem:} Qualquer informação fica privelegiada
		\item Método da transposição: Um registro recuperado com sucesso é trocado imediatamento com o elemento anterior (swap é O(1), não importando a quantidade de elementos). \textbf{Desvantagem:} Cancelamento da otimização, (swap alternados entre mesmos elementos)		
		\end{itemize}
	\item Tabela Ordenada
		\begin{itemize}
		\item Complexidade: O(n/2). \textbf{Pior caso:} Complexidade: O(n)
		\item Dificuldade: Manter tabela ordenada e a ordenação em si
		\end{itemize}
	\item Tabela indexada
		\begin{itemize}
		\item Utilização de tabela auxiliar como tabela de índices
		\item Cada elemento na tabela de índices contém uma chave (kindex) e um indicador do regostro no arquivo que corresponde a kindex
		\end{itemize}
	\item Vantagens e desvantagens na busca sequêncial
		\begin{itemize}
		\item Vantagens: Os ítens poderão ser examinados sem serem acessados, o tempo de busca diminui consideravelmente
		\item Desvantagens: Tabela tenha que estar ordenada, demanda mais espaço.
		\end{itemize}
	\item Remoção
		\begin{itemize}
		\item Remova-se o elemento e rearranja-se a tabela
		\item Indicar que o local está vazio, e futuramente é inserido outro elemento no índice
		\end{itemize}
	\item Inserção
		\begin{itemize}
		\item Se houver espaço vago, rearranjam-se os elementos localmente, caso não haja espaço, toda a tabela deve ser rearranjada
		\end{itemize}
	\end{enumerate}

\end{document}